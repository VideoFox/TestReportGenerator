\documentclass [12pt, a4paper]{report}
\usepackage[left=2cm, right=2cm, top=2.5cm, bottom=3.5cm, bindingoffset=0cm]{geometry}
\usepackage[english]{babel}
\usepackage{fontspec} 
\setmainfont{Arial}
\usepackage{float}
\usepackage{caption}
\captionsetup[table]{position=above,justification=raggedright}
%\captionsetup[figure]{}
\usepackage{tikz}
\usepgflibrary{fpu}
\usepackage{pgfplots}
\usepackage{pgfplotstable}
\pgfplotsset{compat=1.5}

\usepackage{amssymb}
\usepackage{titlesec} %настройка шрифтов оглавлений

\titleformat{\section}{\normalfont\large\bfseries}{\thesection}{1em}{}
\titleformat{\subsection}{\normalfont\normalsize\bfseries}{\thesubsection}{1em}{}

\setlength{\arrayrulewidth}{1pt}%установка ширины линий таблиц

\usepackage{color,colortbl}
\usepackage{color,xcolor}
\usepackage{array}
\usepackage{tabularx}
\usepackage{longtable}
\extrarowheight = 2pt %добавление пространства под верхней линейкой таблицы


\usepackage{graphicx}
\usepackage{multirow}%для возможности объединения строк в таблицах

\usetikzlibrary{plotmarks}

\usepackage[ddmmyyyy]{datetime}
\renewcommand{\dateseparator}{.} 

\usepackage{textcomp} %для добавления знака копирайта
\usepackage{fancyhdr}
\pagestyle{fancy}

\usepackage{lastpage}

\usepackage{ragged2e}
\newcommand{\jj}{\righthyphenmin=20 \justifying} %команда для выравнивания текста по ширине


\headheight = 103pt
\headsep = 8pt
\footskip = 1cm
%\textheight = 21.5cm
\textheight = 22.5cm
\fancyhf{}  % убираем текущие установки для колонтитулов
\renewcommand{\headrulewidth}{0.0pt}
\fancyhead[C]{
\begin{tabularx}{\textwidth}{|>{\raggedright}X|l|l|}
\multicolumn{3}{l}{\large{\TECH\ \MOS\ \CHIPID\ DTC Test Report ~---\ \version ~---\ \today}} \\
\hline
Chip ID: \CHIPID & MPW ID: \MPWID & Wafer No.: \WAFERNO \\
\hline
Batch ID: \BATCHID & Lot ID: \LOTID & Dies: \DIES \\
\hline 
\multicolumn{2}{|l|}{Core lib.: \CORELIBRARY} & IO lib.: \IOLIBRARY \\
\hline
\multicolumn{3}{|l|}{Process: \PROCESS} \\
\hline
\end{tabularx}
}
\fancyfoot[C]{\footnotesize \thepage /\footnotesize \pageref{LastPage}}
\fancyfoot[L]{\footnotesize{\today \ \textcopyright \  X-FAB Semiconductor Foundries}}
\fancyfoot[R]{\footnotesize{\version}}


\fancypagestyle{plain}{
\fancyhf{}  % убираем текущие установки для колонтитулов
\renewcommand{\headrulewidth}{0.0pt}
\fancyhead[C]{
\begin{tabularx}{\textwidth}{|>{\raggedright}X|l|l|}
\multicolumn{3}{l}{\large{\TECH\ \MOS\ \CHIPID\ DTC Test Report ~---\ \version ~---\ \today}} \\
\hline
Chip ID: \CHIPID & MPW ID: \MPWID & Wafer No.: \WAFERNO \\
\hline
Batch ID: \BATCHID & Lot ID: \LOTID & Dies: \DIES \\
\hline 
\multicolumn{2}{|l|}{Core lib.: \CORELIBRARY} & IO lib.: \IOLIBRARY \\
\hline
\multicolumn{3}{|l|}{Process: \PROCESS} \\
\hline
\end{tabularx}
}
\fancyfoot[C]{\footnotesize \thepage /\footnotesize \pageref{LastPage}}
\fancyfoot[L]{\footnotesize{\today \ \textcopyright \  X-FAB Semiconductor Foundries}}
\fancyfoot[R]{\footnotesize{\version}}
}

\usepackage{etoc}
\renewcommand{\etocaftertitlehook}{\pagestyle{plain}}
\renewcommand{\etocaftertochook}{\thispagestyle{plain}}

\setcounter{secnumdepth}{5}
\setcounter{tocdepth}{5}

\usepackage{titletoc}
\usepackage[hidelinks, breaklinks]{hyperref} 
\usepackage{tocloft} 
\dottedcontents{chapter}[1.6em]{}{1.6em}{1pc}
%Изменение стиля страницы с оглавлением.%%%%%%%%%%%%%%%
\setlength{\cftbeforetoctitleskip}{0pt} % отступ перед оглавлением
\setlength{\cftaftertoctitleskip}{0pt} % отступ после оглавления

%Оформление вывода заголовков Chapter%%%%%%%%%%%%%%
\makeatletter
\def\thickhrulefill{\leavevmode \leaders \hrule height 1ex \hfill \kern \z@}
\def\@makechapterhead#1{%
  {\parindent \z@ 
        \reset@font\Large\bfseries
        \begin{tabular}{p{6mm}p{25cm}}
          {\thechapter{}}
          &
          \Large #1
        \end{tabular}
        \par\nobreak
    \vskip 10\p@ %расстояние от названия главы до текста
  }}
%%%%%%%%%%%%%%%%%%%%%%%%%%%%%%%%%%%

%для выравнивания названий таблиц по левому краю%%%%%%%%%%%%%%%%%%%%%%
\newcommand{\@maketablecaption}[2]{
    \vskip\abovecaptionskip
    \sbox\@tempboxa{#1 --- #2}%
    \ifdim \wd\@tempboxa >\hsize
        #1 --- #2\par
    \else
        \global \@minipageFailse
        \hb@xt@\hsize{\box\@tempboxa}%
    \fi
    \vskip\belowcaptionskip}
\renewcommand{\table}{\let\@makecaption\@maketablecaption\@float{table}}
%%%%%%%%%%%%%%%%%%%%%%%%%%%%%%%%%%%%%%%%%%%%%%%